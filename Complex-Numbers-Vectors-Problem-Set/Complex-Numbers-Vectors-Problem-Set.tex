 \documentclass{article}

\usepackage[english]{babel}

\usepackage[letterpaper,top=2cm,bottom=2cm,left=3cm,right=3cm,marginparwidth=1.75cm]{geometry}

\usepackage{amsmath}
\usepackage{graphicx}
\usepackage{amssymb}
\usepackage[colorlinks=true, allcolors=blue]{hyperref}

\title{Complex Numbers / Vectors Problem Set}
\author{Tadhg XG}
\date{2024}

\begin{document}

\maketitle

\textbf{Q1.} (Marist 2022) Let $z = e^{i\theta}, \theta \in (-\pi, \pi].$ 
\begin{enumerate}
    \item[a.] Show that $z^n+z^{-n} = 2 \cos{(n\theta)}.$
    \item[b.] Hence, or otherwise, determine the values of $\theta$ such that,
$$| e^{4i\theta} + 1| = \sqrt{3}.$$
            This can be done geometrically, but an algebraic method is probably easier. You are encouraged to try both!
\end{enumerate}

\textbf{Q2.} Consider some $z_1, z_2 \in \mathbb{C},$ where it is given that $|z_1| = |z_2|$ and $\text{arg}(z_1) < \text{arg}(z_2).$ In the Argand diagram, the origin $O,$ $z_1, z_2$ and $z_1 + z_2$ form a shape.
\begin{enumerate}
    \item[a.] Name the shape, explaining the properties of such shape you used to reach this conclusion.
    \item[b.] Show that the argument of $z_1 + z_2$ is the arithmetic mean of the arguments of our two complex numbers, that is, $$\text{arg}(z_1+z_2) = \frac{\text{arg}(z_1)+\text{arg}(z_2)}{2}. $$
\end{enumerate}

\textbf{Q3.} (BoS 2022) Let $z$ be a complex number such that $|z+i|z||=|z-2i|z||.$ Sketch the locus of $z.$ \\

\textbf{Q4.} Consider the points $A(3,5,-1), B(7,-2,3).$ Denote $\mathcal{L}$ as the line that passes through $A$ and $B.$ Find the shortest distance between the line $\mathcal{L}$ and the origin $O$. \\

\textbf{Q5.} Consider the following two lines based upon the parameters $\lambda, \mu \in \mathbb{R}$,
$$
\vec{r}_1 = 
\begin{pmatrix}
    1 \\
    0 \\
    -1
\end{pmatrix} + \lambda
\begin{pmatrix}
    1 \\
    2 \\ 
    3
\end{pmatrix}, \hspace{0.5cm}
\vec{r}_2 =
\begin{pmatrix}
    -1 \\
    1 \\
    0
\end{pmatrix} + \mu
\begin{pmatrix}
    0 \\
    0 \\
    1
\end{pmatrix}.
$$
Find the shortest distance between the two lines. \\

\textbf{Q6.} (Fitzpatrick) Show that the line between the points $(1, -1, 1$ and $(5,3,3)$ is perpendicular to the line between the points $(1, -1, 2)$ and $(4, -4, 6).$ \\

\textbf{Q7.} Consider the following two lines based upon the parameters $\lambda, \mu \in \mathbb{R}$,
$$
\vec{r}_1 = 
\begin{pmatrix}
    7 \\
    5 \\
    -1
\end{pmatrix} + \lambda
\begin{pmatrix}
    1 \\
    2 \\ 
    3
\end{pmatrix}, \hspace{0.5cm}
\vec{r}_2 =
\begin{pmatrix}
    -1 \\
    1 \\
    4
\end{pmatrix} + \mu
\begin{pmatrix}
    2 \\
    4 \\
    6
\end{pmatrix}.
$$
\begin{enumerate}
    \item[a.] Show that these lines are parallel, and explain why they do not intersect.
    \item[b.] Hence, find the shortest distance between the two lines.
\end{enumerate}


\end{document}
